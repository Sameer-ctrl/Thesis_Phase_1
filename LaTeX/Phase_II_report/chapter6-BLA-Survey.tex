\chapter{Hunt for BLAs : The Survey} \label{ch:Survey}

The foremost goal of the current study is to do a survey of BLAs over a large data-set and then estimate their contribution in the closure density of $\Omega_b$. In this chapter we describe our approach to carry out this exhaustive survey.

\section{Shortlisting the BLA candidates} \label{sec:BLA-candidates}

\citet{danforth-2016} have identified a total of 2611 absorber systems in their study of the low redshift intergalactic medium. We need to find suitable candidates for our survey of BLAs from these 2611 absorbers.

We use two criterias to select BLA candidates for our survey. First, we look for broad Ly$\alpha$ lines in all the 2611 absorbers. For a Ly$\alpha$ line to be adjudged as `broad', we fix our threshold for \emph{b} value to be greater than 45 km s$^{-1}$ in the preliminary fitting done by \citet{danforth-2016}. Assuming a complete thermal broadening, $\emph{b}=45$ km s$^{-1}$ gives a temperature of $\approx 1.2 \times 10^5$ K, which lies in the lower ranges of the temperature of WHIM. By giving this constrain, we get 568 such systems in the complete data-set.

As discussed in section \ref{sec:WHIM-BLA}, multiple lines or contaminations from other lines can blend together to give rise to broad absorption features. In such cases, we need to carefully model these broad lines and confirm that these are actually tracing hot collisionally ionised gas phase so that they are indeed probing WHIM and not just cool photoionised phase. We need to perform ionisation modelling for it. To model the ionisation conditions of the absorber systems, we need metal ion column densities. So we search for systems showing metal line absorption in these 568 `broad' systems. We need at least three distinct metal ions (not lines) to better constrain the ionisation state of an absorber system. This sets our second criteria that there should be minimum of three metal ion absorption in the absorber systems. Upon putting this constrain, we get 29 systems having at least 3 distinct metal ions out of 568 already identified systems. Out of these 29 systems, we have already studied one of the absorber in chapter \ref{ch:PG0003}. Table \ref{tab:BLA-candidates} lists the lines of sight, redshift of the absorber and the ions detected in the systems these 29 identified BLA candidate absorber systems. Figure \ref{fig:metal-ions-hist} shows the distribution of different metal ions found in these 29 absorber systems. These 29 absorbers are found along 22 different sight lines. The details these lines of sight are given in appendix \ref{ap:los}. The details of the HST/COS observations of these 22 sight lines are given in appendix \ref{ap:COS-observations}


\begin{figure}
    \centering
    \includegraphics[width=\linewidth]{Figures/metal-ions.png}
    \caption{No. of different metal ions in all the 29 absorber systems.}
    \label{fig:metal-ions-hist}
\end{figure}


\begin{table}
\centering
\vspace{5mm}
\hspace*{-15mm}
    \begin{tabular}{cccc}
        \hline \hline
       \head{S. no.} & \head{Sight line} & \head{$\mathbf{z_{abs}}$} &  \head{Metal ions}
       \tabularnewline \hline  \tabularnewline
1    &    1ES 1553+113   &   0.187731   &  \ion{C}{iii}, \ion{O}{vi}, \ion{N}{v}   \\ 
2    &    3C 263   &   0.063275   &  \ion{C}{iv}, \ion{Si}{iii}, \ion{Si}{iv}   \\ 
3    &    3C 263   &   0.140754   &  \ion{C}{iv}, \ion{Si}{iii}, \ion{O}{vi}   \\ 
4    &    3C 57   &   0.077493   &  \ion{C}{iv}, \ion{Si}{iv}, \ion{N}{v}   \\ 
5    &    H 1821+643   &   0.170062   &  \ion{Si}{iii}, \ion{O}{vi}, \ion{N}{v}   \\ 
6    &    H 1821+643   &   0.224832   &  \ion{Si}{iii}, \ion{O}{vi}, \ion{C}{iii}   \\ 
7    &    HE 0056-3622   &   0.043318   &  \ion{C}{iv}, \ion{Si}{iii}, \ion{N}{v}   \\ 
8    &    PMN J1103-2329   &   0.003975   &  \ion{C}{iv}, \ion{Si}{iii}, \ion{Si}{iv}, \ion{N}{v}   \\ 
9    &    PG 0003+158   &   0.386094   &  \ion{C}{iii}, \ion{O}{vi}, \ion{O}{iii}, \ion{N}{v}   \\ 
10    &    PG 0003+158   &   0.347586   &  \ion{C}{ii}, \ion{C}{iii}, \ion{Si}{ii}, \ion{Si}{iii}, \ion{O}{vi}\\ 
11    &    PG 0003+158   &   0.421880   &  \ion{O}{vi}, \ion{O}{iii}, \ion{C}{iii}   \\ 
12    &    PG 0832+251   &   0.017520   &  \ion{C}{iv}, \ion{Si}{iv}, \ion{O}{i}, \ion{Si}{iii}, \ion{C}{ii}, \ion{Si}{ii}, \ion{Fe}{ii}, \ion{Al}{ii}, \ion{N}{v}   \\ 
13    &    PG 1116+215   &   0.138527   &  \ion{C}{iv}, \ion{Si}{iv}, \ion{N}{ii}, \ion{P}{ii}, \ion{Si}{iii}, \ion{Si}{ii}, \ion{C}{ii}, \ion{O}{vi}, \ion{N}{v}   \\ 
14    &    PG 1121+422   &   0.192434   &  \ion{Si}{iv}, \ion{C}{iii}, \ion{Si}{iii}, \ion{Si}{ii}, \ion{C}{ii}, \ion{O}{vi}   \\ 
15    &    PG 1216+069   &   0.006390   &  \ion{O}{i}, \ion{Si}{ii}, \ion{C}{ii}   \\ 
16    &    PG 1216+069   &   0.282195   &  \ion{Si}{iii}, \ion{O}{vi}, \ion{C}{iii}   \\ 
17    &    PG 1222+216   &   0.054491   &  \ion{C}{iv}, \ion{Si}{iii}, \ion{Si}{iv}   \\ 
18    &    PG 1222+216   &   0.378600   &  \ion{Si}{iii}, \ion{O}{vi}, \ion{O}{iii}, \ion{C}{iii}   \\ 
19    &    PG 1259+593   &   0.046107   &  \ion{C}{iv}, \ion{Si}{iii}, \ion{Si}{iv}   \\ 
20    &    PG 1424+240   &   0.146789   &  \ion{C}{iv}, \ion{Si}{iii}, \ion{O}{vi}, \ion{Si}{iv}   \\ 
21    &    PHL 1811   &   0.080837   &  \ion{C}{iv}, \ion{Si}{iv}, \ion{N}{ii}, \ion{O}{i}, \ion{Fe}{ii}, \ion{Si}{ii}, \ion{C}{ii}   \\ 
22    &    PKS 0405-123   &   0.167125   &  \ion{Si}{iv}, \ion{N}{ii}, \ion{C}{iii}, \ion{O}{i}, \ion{Si}{iii}, \ion{Si}{ii}, \ion{C}{ii}, \ion{O}{vi}, \ion{N}{iii}, \ion{N}{v}   \\ 
23    &    PKS 0637-752   &   0.161068   &  \ion{Si}{iii}, \ion{O}{vi}, \ion{N}{v}   \\  
24    &    PKS 0637-752   &   0.417573   &  \ion{Si}{iii}, \ion{O}{vi}, \ion{C}{iii}   \\ 
25    &    PKS 1302-102   &   0.094864   &  \ion{Si}{iii}, \ion{Si}{ii}, \ion{C}{ii}   \\ 
26    &    RX J0439.6-5311   &   0.005602   &  \ion{C}{iv}, \ion{Si}{iii}, \ion{Si}{iv}   \\ 
27    &    SDSS J135712.61+170444   &   0.097767   &  \ion{C}{iv}, \ion{Si}{iv}, \ion{Si}{iii}, \ion{C}{ii}, \ion{O}{vi}   \\ 
28    &    SBS 1108+560   &   0.463201   &  \ion{C}{iii}, \ion{O}{i}, \ion{Si}{iii}, \ion{Si}{ii}, \ion{C}{ii}, \ion{O}{vi}, \ion{N}{iii}   \\ 
29    &    UKS 0242-724   &   0.063775   &  \ion{Fe}{ii}, \ion{Si}{ii}, \ion{C}{ii}   \\  
       \tabularnewline \hline \hline 
    \end{tabular}
\caption{Details of the 29 BLA candidate absorber system shortlisted  for the survey.}
\label{tab:BLA-candidates}
\end{table}


\section{Survey methodology}

We have identified 28 additional BLA candidate systems for our survey. We need to do the Voigt profile fitting to the absorption lines identified in these systems. These will give us the column densities and Doppler parameters of the ions in the system and also their redshifts (velocities). We will further use these quantities to model the ionisation conditions in these absorber systems. 

The distribution of these quantities can give valuable insights towards our understanding of the intergalactic medium and the baryon content within IGM. As discussed in chapter \ref{chap:intro}, that \ion{O}{vi} is good tracer of WHIM. \ion{O}{vi} absorption is seen in 16 out of these remaining 16 absorbers. For the remaining 12 candidates, \ion{O}{vi} is not a non-detection. The \ion{O}{vi} 1032, 1038 lines fall out the coverage of the HST/COS FUV channel at redshifts below $\sim 0.093854$ and all these remaining 12 systems are at redshift below 0.093854. So \ion{O}{vi} is not covered in these systems. However, for one system which is along the LOS of PKS1302-102 at $z_{abs}=0.094864$, the \ion{O}{vi} 1038 line was just falling just on the edge of the spectrum where both S/N and sensitivity both low. So, \ion{O}{vi} absorption was not considered for this system. 

We first model the ionisation conditions in the 16 \ion{O}{vi} absorbers and see if we can explain the origin of \ion{O}{vi} through photoionisation models using the similar method used for the absorber in chapter \ref{ch:PG0003}. For, the remaining 12 non-\ion{O}{vi} absorber, we model the ionisation conditions based on the ions detected in the systems to estimate the density and metallicity in these systems. 

Then, we will use the results from this survey to estimate the baryon content in BLAs and their contribution to cosmic closure density, $\Omega_b$, the details of which are described in the upcoming chapter.

The Voigt profile fitting and ionisation modelling results are given in appendix \ref{ap:survey-results} after the references.


\section{Survey statistics}

In this section, we discuss the statistics and results of the survey from Voigt profile fitting and ionisation modelling of the 29 absorber systems.

\subsection{Voigt profile fitting}

This survey of 29 absorbers spanned 22 lines of sight. These 22 lines of sight have total \ion{H}{i} (Lyman-$\alpha$) redshift pathlength of $\Delta z = 5.561$. A total of 413 absorption lines were identified and fitted with Voigt profiles. These 413 lines shown absorption from 15 different metal ions apart from \ion{H}{i} absorption. The figure \ref{fig:comp-distribution} shows the total number of components identified for each species. For studies similar to current one, it is very insightful to describe the line density per unit redshift ($d\mathcal{N}/dz$) and  bivariate distribution of the absorbers with respect to column density and redshift ($\partial ^2 \mathcal{N}/\partial N \ \partial z $) (see e.g. \citet{danforth-2016, Penton-2000, tilton_2012}). However, since our current sample is not statistically large, we don't describe these metrics while discussing the statistics of the survey.

\begin{figure}
    \centering
    \includegraphics[width=\linewidth]{Figures/ion-component.png}
    \caption{No. of components identified for \ion{H}{i} and different metal ions.}
    \label{fig:comp-distribution}
\end{figure}

\subsubsection{H \hspace*{-0.5mm}{\footnotesize I} absorbers}

We have identified 97 \ion{H}{i} components in our survey across 29 absorbers. The figure \ref{fig:HI_distribution} shows the distribution of column densities and redshift of these 97 components. These components span seven orders of magnitude of column densities starting from $\sim 10^{13} \ \text{to} \ 10^{19} \ \text{cm}^{-2}$. We have many absorbers with the column densities in the range $10^{13} - 10^{15.5} \ \text{cm}^{-2}$ but current sample is incomplete at rare large column density systems. In our current sample, we see \ion{H}{i} absorption till $z \sim 0.45$. In redshift space also, we have more systems at lower redshifts ($z < 0.2$) and only few absorbers at higher redshifts.

\begin{figure}
    \centering
    \includegraphics[width=\linewidth]{Figures/HI_distribution_survey.png}
    \caption{Distribution of column densities (left panel) and redshift (right panel) of 97 \ion{H}{i} components.}
    \label{fig:HI_distribution}
\end{figure}

\subsubsection*{Distribution of Doppler parameters}

The figure \ref{fig:b_HI_distribution} shows the distribution of Doppler \emph{b} parameters of the 97 \ion{H}{i} components. The vertical dashed line marks the cutoff for the line to be called as `BLA' at \emph{b}=40 km $\text{s}^{-1}$. Assuming a complete thermal broadening, $\emph{b}=40$ km s$^{-1}$ gives a temperature of $10^5$ K. Out of 97 components, 37(34) components have Doppler parameters in excess of 40(45) km $\text{s}^{-1}$. Some components show large Doppler widths of above 100 km $\text{s}^{-1}$, for these components only saturated Ly$\alpha$ lines are seen, making the identification of velocity substructures of these lines difficult. This results in the large \emph{b} values for these lines. 

\begin{figure}
    \centering
    \includegraphics[width=\linewidth]{Figures/b_HI_distribution_survey.png}
    \caption{Distribution of \emph{b} parameters of 97 \ion{H}{i} components.}
    \label{fig:b_HI_distribution} 
\end{figure}

\subsubsection*{Temperature estimation using Doppler parameters}  \label{sec:Temp}

We make an estimate of the temperature of the absorbers directly from their Doppler parameters as done for the absorber in chapter \ref{ch:PG0003}. To do so we need the \ion{H}{i} lines to be fairly aligned with some other metal ion. Because, if the species are aligned, then we can asssume that they are co-spatial, so that there non-thermal broadening could be assumed to be same. However, we need to be cautious that aligned species need not be co-spatial always. HST/COS data may have wavelength calibration errors of 15-20 km s$^{-1}$, which could be as high as few resolution elements \citep{Wakker-2015}. So, we consider species as aligned if they are separated by less than 20 km s$^{-1}$ in velocity within the fitting errors. We use \ion{O}{vi} as the metal specie for estimating the temperature.

We plot the \emph{b}(\ion{O}{vi}) vs \emph{b}(\ion{H}{i}) in figure \ref{fig:bOVI_bHI} for the 17 absorbers. The orange dashed line corresponds to points where \emph{b}(\ion{H}{i}) = 4\emph{b}(\ion{O}{vi}) which is the case for pure thermal broadening, i.e. non-thermal contributions are zero. The blue marks the region where the non-thermal broadening dominates the line broadening. This is \emph{b}(\ion{H}{i}) = \emph{b}(\ion{O}{vi}) line. And vertical green dashed line marks our cutoff for the line to be called as BLA. Now, we can estimate the temperature for a system if the \emph{b} values of these two species lies within these orange and blue lines given that they are aligned.  

We could estimate the temperature for 7 components where \ion{H}{i} and \ion{O}{vi} were aligned. We list the details of these systems in table \ref{tab:Temperature}. This estimation of temperature is important in our estimation of $\Omega_b(\text{BLA})$ (see section \ref{sec:sampleA}).

\begin{figure}
    \centering
    \includegraphics[width=\linewidth]{Figures/bHi_vs_BOvi.png}
    \caption{\emph{b}(\ion{O}{vi}) vs \emph{b}(\ion{H}{i}) plot. The orange dashed line is the \emph{b}(\ion{H}{i}) = 4\emph{b}(\ion{O}{vi}) line and blue dashed line shows the \emph{b}(\ion{H}{i}) = \emph{b}(\ion{O}{vi}) line.}
    \label{fig:bOVI_bHI} 
\end{figure}



\begin{table}
    \centering
    % \vspace*{-2cm}
    % \hspace{-12mm}
    \begin{tabular}{c@{\hspace{1.2em}}c@{\hspace{1em}}c@{\hspace{1em}}c@{\hspace{1em}}c@{\hspace{1em}}c@{\hspace{1em}}c@{\hspace{1em}}}
    \hline \hline 
    \head{Sight line} & \head{$\mathbf{z_{\text{abs}}}$} & \head{$\mathbf{\Delta v}$}\textsuperscript{a} & \head{\emph{b}(H \hspace*{-0.5mm}{\footnotesize I})} & \head{\emph{b}(O \hspace*{-0.5mm}{\footnotesize VI})}  & \head{$\mathbf{b_{\text{th}}}$} & \head{$\mathbf{\text{log T}}$} \\ 

         &  &  (km s$^{-1}$) &  (km s$^{-1}$)  &  (km s$^{-1}$)  &  (km s$^{-1}$)  &  (K) \\

    \hline \tabularnewline

    PG 0003+158 & 0.347586 & 0 $\pm$ 1 & $62 \pm 3$ & $30 \pm 2$ & $56 \pm 2$ & 5.28 $\pm$ 0.05  \\
    PG 0003+158 & 0.386089 & 14 $\pm$ 9 & $40 \pm 4$ & $25 \pm 4$  & $32 \pm 3$ & 4.80 $\pm$ 0.11\\
    1ES 1553+113 & 0.187764 & 28 $\pm$ 1\textsuperscript{b}  & $51 \pm 1$ & $15 \pm 3$ & $50 \pm 1$ & 5.19 $\pm$ 0.04  \\
    PG 1222+216 & 0.378389 & 4 $\pm$ 15 & $52 \pm 4$ & $34 \pm 13$ & $41 \pm 6$ & 5.00 $\pm$ 0.17  \\
    PG 1222+216 & 0.378389 & 18 $\pm$ 4  & $43 \pm 1$ & $29 \pm 13$ & $33 \pm 6$ & 4.81 $\pm$ 0.19 \\
    PG 1116+215  & 0.138527 & 4 $\pm$ 9 & $71 \pm 14$ & $35 \pm 3$ & $64 \pm 8$ & 5.39 $\pm$ 0.21  \\
    H 1821+643 & 0.224981 & 19 $\pm$ 10 & $84 \pm 13$ & $45 \pm 1$ & $73 \pm 8$ & 5.51 $\pm$ 0.16  \\
        \tabularnewline
    \hline \hline 
    \multicolumn{7}{l}{\footnotesize{\textsuperscript{a} Velocity separation between \ion{H}{i} and \ion{O}{vi} components}} \\
    \multicolumn{7}{l}{\footnotesize{\textsuperscript{b} Even though $\Delta v > 20$ km s$^{-1}$ we still assume it to be aligned.}}
    \end{tabular}
    \caption{Details of temperature estimation from Doppler parameters.}
    \label{tab:Temperature}
    \end{table} 

\subsubsection*{N(H \hspace*{-0.6mm}{\footnotesize I})-\emph{b}(H \hspace*{-0.6mm}{\footnotesize I}) distribution} 

We plot the \ion{H}{i} column density with Doppler parameter in figure \ref{fig:NHi_bHi}. We observe that lines with large Doppler widths are found within narrow range of column densities towards the lower end. It could possibly be because large \emph{b} values indicates the presence of high temperature gas, where neutral fraction of Hydrogen will be small. We also see that the lines with large \emph{b} values have large uncertainities also. It is because that most of these are coming from absorbers at lower redshifts where only Ly$\alpha$ is covered in COS/FUV, which is saturated. These large uncertainities affect our estimation of $\Omega_b\text{(BLA)}$ value as discussed in next chapter (see section \ref{sec:BLA-sample}). 

\begin{figure}[!h]
    \centering
    \includegraphics[width=\linewidth]{Figures/NHi_vs_bHi.png}
    \caption{N(\ion{H}{i}) vs \emph{b}(\ion{H}{i}) plot for the 97 \ion{H}{i} components. The vertical black dashed line denotes our threshold for line to be considered as BLA.}
    \label{fig:NHi_bHi}
\end{figure}

\subsubsection{Metal absorbers}

We have identified absorption from total of 15 distinct metal ions in our survey. In this section we discuss the statistics of some prominent ions in our sample.

\subsubsection*{O \hspace*{-0.5mm}{\footnotesize VI}}

We identified 35 \ion{O}{vi} components in our survey across 17 absorber systems. \ion{O}{vi} is the most common ion after \ion{Si}{iii} (36 components) in our sample, owing to the strong doublet lines at $\lambda_\text{rest}=1031.927, \ 1037.616$  \AA, with fairly large oscillator strengths of 0.1329 and 0.0661 respectively and large cosmic abundance of Oxygen. The figure \ref{fig:OVI_distribution} shows the distribution of column densities and redshifts of these 35 components. The \ion{O}{vi} coverage starts from $z \sim 0.09$ in the HST/COS FUV G130M grating. Unlike \ion{H}{i}, we find \ion{O}{vi} columnd densities in narrow range of $\sim 10^{13} - 10^{14.5} \ \text{cm}^{-2}$. 

\begin{figure}
    \centering
    \includegraphics[width=\linewidth]{Figures/OVI_distribution_survey.png}
    \caption{Distribution of column densities (left panel) and redshift (right panel) of 35 \ion{O}{vi} components.}
    \label{fig:OVI_distribution}
\end{figure}

\ion{O}{vi} is a good tracer of WHIM at temperatures in range of $10^5 - 10^6$ K as its ionisation fraction peaks at around $10^{5.7}$ K in collisional ionisation \citep{Gnat-Sternberg-2007}. So, many studies of \ion{O}{vi} absorbers have been done in detail in the past. \citet{savage-2014} have studied 14 QSO sight lines comprising a total of 56 \ion{O}{vi} absorbers showing 85 \ion{O}{vi} components. They estimate the baryonic content in warm gas traced by \ion{O}{vi} as $\Omega_b(\ion{O}{vi})_\text{Warm}= (0.0019 \pm 0.0005) h^{-1}_{70}$.

Studies exploring the correlations between star formation and \ion{O}{vi} have also been done. In their study, \citet{Tumlinson-2011,Tumlinson-2013} found strong correlation between the specific star formation rate and the presence of \ion{O}{vi}. They found that star-forming galaxies are surrounded by large halos ionised \ion{O}{vi} of sizes $\sim 150$ kpc, which were not much prominent around galaxies with little to no star-formation.


\subsubsection*{Si \hspace*{-0.5mm}{\footnotesize II}, Si \hspace*{-0.5mm}{\footnotesize III} and Si \hspace*{-0.5mm}{\footnotesize IV}}

We find good number of Si absorbers in the form of \ion{Si}{ii}, \ion{Si}{iii} and \ion{Si}{iv} ions having 19, 36 and 20 components respectively. \ion{Si}{ii} shows a number of lines in the COS/FUV channel with $\lambda_\text{rest}=1526.707, 1304.371, 1260.422$ \AA \  and doublet at 1190.416, 1193.289 \AA. \ion{Si}{iii} has the most components among the metal ions in our sample because of very high oscillator strength of 1.669 at $\lambda_\text{rest}=1206.5$ \AA. And \ion{Si}{iv} shows doublet line at $\lambda_\text{rest}=1393.760,1402.772$ \AA. Figures \ref{fig:SiII_distribution}, \ref{fig:SiIII_distribution} and \ref{fig:SiIV_distribution} shows the distribution of column densities and redshifts of \ion{Si}{ii}, \ion{Si}{iii} and \ion{Si}{iv} respectively.


\begin{figure}
    \centering
    \includegraphics[width=\linewidth]{Figures/SiII_distribution_survey.png}
    \caption{Distribution of column densities (left panel) and redshift (right panel) of 19 \ion{Si}{ii} components.}
    \label{fig:SiII_distribution}
\end{figure}

\begin{figure}
    \centering
    \includegraphics[width=\linewidth]{Figures/SiIII_distribution_survey.png}
    \caption{Distribution of column densities (left panel) and redshift (right panel) of 36 \ion{Si}{iii} components.}
    \label{fig:SiIII_distribution}
\end{figure}

\begin{figure}
    \centering
    \includegraphics[width=\linewidth]{Figures/SiIV_distribution_survey.png}
    \caption{Distribution of column densities (left panel) and redshift (right panel) of 20 \ion{Si}{iv} components.}
    \label{fig:SiIV_distribution}
\end{figure}

\subsubsection*{C \hspace*{-0.5mm}{\footnotesize II}, C \hspace*{-0.5mm}{\footnotesize III} and C \hspace*{-0.5mm}{\footnotesize IV}}

\ion{C}{ii}, \ion{C}{iii} and \ion{C}{iv} are some other common ions found in our survey which shows 23, 26 and 22 components respectively. Absorption from \ion{C}{ii} is from two transitions with $\lambda_\text{rest}=1036.3367, 1334.5323$ \AA . The \ion{C}{iii} line with $\lambda_\text{rest}=977.0201$ \AA \ has a high oscillator strength of 0.757 which results in prominent absorption from this line, hence it is very common ion found in IGM. \ion{C}{iv} shows the absorption in the form doublet lines with $\lambda_\text{rest}=1548.2041, 1550.7812$ \AA \. Figures \ref{fig:CII_distribution}, \ref{fig:CIII_distribution} and \ref{fig:CIV_distribution} shows the distribution of column densities and redshifts of \ion{C}{ii}, \ion{C}{iii} and \ion{C}{iv} respectively. All the three ions have narrow ranges of column densities with a very few exceptional rare high column density components. We can see in the figure \ref{fig:CIII_distribution} that \ion{C}{iii} can be observed even at lower column densities of $\sim 10^{12.5} \ \text{cm}^{-2}$ because of the large oscillator strength \ion{C}{iii} 977 transition. \ion{C}{iv} doublet lines have a limited coverage in the HST/COS FUV channel as they fall out of the COS coverage at $z \gtrsim 0.15$ which could be seen in figure \ref{fig:CIV_distribution}.



\begin{figure}
    \centering
    \includegraphics[width=\linewidth]{Figures/CII_distribution_survey.png}
    \caption{Distribution of column densities (left panel) and redshift (right panel) of 23 \ion{C}{ii} components.}
    \label{fig:CII_distribution}
\end{figure}

\begin{figure}
    \centering
    \includegraphics[width=\linewidth]{Figures/CIII_distribution_survey.png}
    \caption{Distribution of column densities (left panel) and redshift (right panel) of 26 \ion{C}{iii} components.}
    \label{fig:CIII_distribution}
\end{figure}

\begin{figure}
    \centering
    \includegraphics[width=\linewidth]{Figures/CIV_distribution_survey.png}
    \caption{Distribution of column densities (left panel) and redshift (right panel) of 22 \ion{C}{iv} components.}
    \label{fig:CIV_distribution}
\end{figure}


\subsubsection*{N \hspace*{-0.5mm}{\footnotesize V}}

We find only 14 \ion{N}{v} components in our current survey, which could be attributed to the lower cosmic abundance of Nitrogen compared to rest of the prominent metals detected in our survey. It shows absorption from the doublet lines at $\lambda_\text{rest}=1238.821, 1242.804$ \AA. Figures \ref{fig:NV_distribution} shows the distribution of column densities and redshifts of \ion{N}{v} components. It could also trace gas with high temperature above $10^5$ K and could also arise from photoionisation from highly energetic photons with energies in the range of $\sim 50-100$ ev.


\begin{figure}
    \centering
    \includegraphics[width=\linewidth]{Figures/NV_distribution_survey.png}
    \caption{Distribution of column densities (left panel) and redshift (right panel) of 14 \ion{N}{v} components.}
    \label{fig:NV_distribution}
\end{figure}



\subsection{Ionisation modelling}  \label{sec:Ionisation-Modelling-statistics}

We have done the ionisation modelling for a total of 39 components in 29 absorbers. Out of this, 25 components are from \ion{O}{vi} absorbers and remaining are from non-\ion{O}{vi} absorbers. We present the results from ionisation modelling of all these components in this section.

From ionisation modelling, we get the Hydrogen density\footnote{We use this interchangeably with absorber density, i.e density at that component} $(n_H)$ and metallicity (Z) of the absorbers, which dictates the prevalent ionising and physical conditions in the absorber clouds. The figure \ref{fig:nH-Z} shows these values for all the 39 components plotted against each other. We don't see any correlations between the two quantities which is expected as these independently describe the physical conditions of the absorbers. The \ion{O}{vi} absorbers are denoted with red and non-\ion{O}{vi} absorbers are marked by green error bars. 


\begin{figure}[!t]
    \centering
    \includegraphics[width=\linewidth]{Figures/Z_vs_nH.png}
    \caption{Metallicity vs density for all 39 components estimated from ionisation modelling. Red error bars are the \ion{O}{vi} components and green error bars are non-\ion{O}{vi} components}
    \label{fig:nH-Z}
\end{figure}

However, when we see the variation of these two quantities with the underlying neutral Hydrogen column density (N(\ion{H}{i})) in these components, we see some trends. Figure \ref{fig:nH-NHi} shows the variation of density with N(\ion{H}{i}). We see a feeble positive relation between $n_H$ and N(\ion{H}{i}) for the components from \ion{O}{vi} absorbers. This is expected as higher column density would typically result in higher densities if the sizes of absorbers, ionisation corrections are of similar order. The lack of such correlation in the non-\ion{O}{vi} case may be due to less number of components.

\begin{figure}[!htbp]
    \centering
    \includegraphics[width=0.9\linewidth]{Figures/nH_vs_NHi.png}
    \caption{Variation of density with N(\ion{H}{i}) for all 39 components. Red error bars are from \ion{O}{vi} absorbers and green error are from non-\ion{O}{vi} absorbers}
    \label{fig:nH-NHi}
\end{figure}

Unlike density, we find a negative correlation between the metallicity and N(\ion{H}{i}) as shown in figure \ref{fig:Z-NHi} which could be seen for both \ion{O}{vi} and non-\ion{O}{vi} absorbers. However, again due to less number of components for non-\ion{O}{vi} absorbers, it less evident in them. This could be explained by the fact that as we get larger N(\ion{H}{i}), the column densities of metals do not scale accordingly with N(\ion{H}{i}), they remain within the narrow ranges of their distribution. So to recover large N(\ion{H}{i}), large line of sight thickness has to integrated with nearly same amount of metals, which results in the drop in metallicity.


\begin{figure}[!htbp]
    \centering
    \includegraphics[width=0.9\linewidth]{Figures/Z_vs_NHi.png}
    \caption{Variation of metallicity with N(\ion{H}{i}) for all 39 components. Red error bars are the \ion{O}{vi} components and green error bars are non-\ion{O}{vi} components}
    \label{fig:Z-NHi}
\end{figure}


\subsubsection{Origin of O \hspace*{-0.5mm}{\footnotesize VI} in the absorbers}

The 17 \ion{O}{vi} absorbers have immense importance in our current study so are there ionisation conditions. Based on the results of these 17 absorbers, we estimate the $\Omega_b(\text{BLA})$ value as discussed in next chapter. For these 17 systems, having 25 components, we want to find the origin of \ion{O}{vi} in these absorbers so that we could see if they are tracing warm-hot gas or cool photoionised gas phase. So, if they arise from a warm-hot plasma, we could infer that the BLA candidate found with \ion{O}{vi} could also be thermally broadend. Figure \ref{fig:OVI-origin} shows the bar plot of the inferred origin of these 25 components. 20 of the components could not be explained with photoionisation (PI) models, so possibly tracing a collisional ionised (CI) phase. One of the components, shown agreement with PI models and in 4 components the origin of \ion{O}{vi} remained uncertain due to models failing to predict the column densities of other ions detected. The absorber which has this PI component also shows CI origin in another component. Out of the 4 uncertain components, two of them do not show good solution, however, in other two components \ion{O}{vi} could be tentatively collisionally ionised, the bad solution is due to large number of ions present in the absorbers where our models fail considerably. However, to be conservative we do not count them in CI case. Table \ref{tab:OVI-origin} gives the details of all these 25 components.

Figure \ref{fig:ex-CI} shows an example of CI origin of \ion{O}{vi}, where other ions could be explained with PI models but not \ion{O}{vi} (orange color). Figure \ref{fig:ex-PI} shows the only example of PI origin of \ion{O}{vi}, where we get similar solution for excluding and including \ion{O}{vi} cases, indicating that all ions could be explained with PI models. Figure \ref{fig:ex-uncertain} shows uncertain case, where our model fails to predict the column densities of other ions as well. 

Ionisation modelling of these absorbers shows that the gas in these absorbers, which are possibly tracing WHIM, is multiphase in nature, having ions arising from photoionisation as well as collisional ionisation. In the next chapter, we use these results from ionisation modelling as well as Voigt profile fitting to estimate the baryon content in BLAs.  


\begin{figure}
    \centering
    \includegraphics[width=0.9\linewidth]{Figures/OVI_cases.png}
    \caption{Origin of \ion{O}{vi} in 25 components from 17 \ion{O}{vi} absorbers.}
    \label{fig:OVI-origin}
\end{figure}


\begin{figure}
    \centering
    \includegraphics[width=0.9\linewidth]{Figures/s135712-z=0.097869-compII.png}
    \caption{An example of CI case for an absorber towards the line of sight of SDSS J135712.61+170444 at $z_{abs}=0.097869$ with log N(\ion{H}{i}) [cm\textsuperscript{-2}] = 16.49}
    \label{fig:ex-CI}
\end{figure}

\begin{figure}
    \centering
    \includegraphics[width=0.9\linewidth]{Figures/1es1553-z=0.187764-compI.png}
    \caption{Only example of PI case for an absorber towards the line of sight of 1ES 1553+113 at $z_{abs}=0.187764$ with log N(\ion{H}{i}) [cm\textsuperscript{-2}] = 12.76}
    \label{fig:ex-PI}
\end{figure}


\begin{figure}
    \centering
    \includegraphics[width=0.9\linewidth]{Figures/pks0405-z=0.167125-compII.png}
    \caption{An example of uncertain case for an absorber towards the line of sight of PKS 0405-123 at $z_{abs}=0.167125$ with log N(\ion{H}{i}) [cm\textsuperscript{-2}] = 13.46}
    \label{fig:ex-uncertain}
\end{figure}


\begin{table}[!h]
    \centering
    \vspace{5mm}
        \begin{tabular}{cccc}
            \hline \hline
           \head{Sight line} & \head{$\mathbf{z_{abs}}$} &  \head{log N(H \hspace*{-0.5mm}{\footnotesize I})}  &  \head{Origin of O \hspace*{-0.5mm}{\footnotesize VI}}   \tabularnewline
           
            &  &  (cm\textsuperscript{-2})  &   \tabularnewline \hline 

        3C 263  &  0.140756  &  14.49  &  CI \\
        PKS 0637-752  &  0.161064  &  13.60  & CI  \\
        PKS 0637-752  &  0.417539  &  15.41  & CI  \\
        PG 1424+240  &  0.147104  &  14.88  &  CI \\
        PG 1424+240  &  0.147104  &  15.44  &  CI \\
        PG 0003+158  &  0.347586  &  16.10  &  CI \\
        PG 0003+158  &  0.386089  &  14.81  &  uncertain\textsuperscript{a} \\
        PG 0003+158  &  0.421923  &  14.17  &  CI \\
        PG 1216+069  &  0.282286  &  16.40  &  CI \\
        SDSS J135712.61+170444  &  0.097869  &  15.01  &  CI \\
        SDSS J135712.61+170444  &  0.097869  &  16.49  &  CI \\
        1ES 1553+113  &  0.187764  &  12.76  &  PI \\
        1ES 1553+113  &  0.187764  &  13.88  &  CI \\
        SBS 1108+560  &  0.463207  &  15.79  &  CI \\
        SBS 1108+560  &  0.463207  &  18.10  & uncertain\textsuperscript{b}  \\
        PG 1222+216  &  0.378389  &  15.43  &  CI \\
        PG 1116+215  &  0.138527  &  13.60  &  uncertain\textsuperscript{b} \\
        H 1821+643  &  0.170006  &  13.35  &  CI \\
        H 1821+643  &  0.170006  &  13.68  &  CI \\
        H 1821+643  &  0.224981  &  15.13  &  CI \\
        H 1821+643  &  0.224981  &  15.16  &  CI \\
        PG 1121+422  &  0.192393  &  14.34  &  CI \\
        PG 1121+422  &  0.192393  &  17.70  &  CI \\
        PKS 0405-123  &  0.167125  &  13.46  & uncertain\textsuperscript{a}  \\
        PKS 0405-123  &  0.167125  &  15.98  & CI  \\

        \hline \hline  
        \multicolumn{4}{l}{\textsuperscript{a} \footnotesize{Other ions also could not be explained}} \\     
        \multicolumn{4}{l}{\textsuperscript{b} \footnotesize{Bad solution due to many ions}} \\ 

        \end{tabular}
    \caption{25 \ion{O}{vi} components and origin of \ion{O}{vi} in them}
    \label{tab:OVI-origin}
\end{table}

