\chapter*{Abstract}

Baryons have an intricate structure in our present universe ($z\sim 0$). Around 80\% of them are residing outside the collapsed structures of galaxies, groups and galaxy clusters in the intergalactic medium (IGM). Even in IGM they exist in multi-phase structure, viz in cool (T$\sim10^4$ K) photoionised phase in the form of Lyman-$\alpha$ forests, constituting about 30\% of the total baryon budget and other 50\% in the warm-hot phase called Warm Hot Intergalactic Medium (WHIM) with temperature in the ranges of $10^5-10^7$ K. The WHIM has been challenging to observe leading to uncertainties in the baryon census in this phase. 

In the present work we aim address these uncertainties. We use high S/N HST/COS data to probe WHIM using Broad Lyman-$\alpha$ Absorbers (BLAs). In the first part of the work we study an interesting absorber system towards the line of sight of quasar PG 0003+158, which has a BLA candidate. We first fit the Voigt profiles to the absorption lines in the absorber system and then model the ionisation conditions in the absorber cloud to infer its ionisation state. Along with studying the galaxy neighbourhood of the absorber system we infer that the absorber is residing in a galaxy under-dense region and could be tracing a reservoir of baryons in a large scale filamentary structure in the cosmic web or CGM of a sub-$\text{L}^{*}$ galaxy.

The next part of the work includes the survey of BLAs over a large data-set. We present partial results and updates on the survey and conclude with future plan of action.
