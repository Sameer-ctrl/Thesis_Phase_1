\chapter*{Abstract}

Baryons have an intricate structure in our present universe ($z\sim 0$). Around 80\% of them are residing outside the collapsed structures of galaxies, groups and galaxy clusters in the intergalactic medium (IGM). Even in IGM they exist in multi-phase structure, viz in cool (T$\sim10^4$ K) photoionised phase in the form of Lyman-$\alpha$ forests, constituting about 30\% of the total baryon budget and other 50\% in the warm-hot phase called Warm Hot Intergalactic Medium (WHIM) with temperature in the ranges of $10^5-10^7$ K. The WHIM has been challenging to observe leading to uncertainties in the baryon census in this phase. This has led to `the missing baryon problem' in the current universe which states that around 30\% of the baryons in present universe are missing from our observations.

In the present work we aim to address these uncertainties. We use high S/N HST/COS data to probe WHIM using Broad Lyman-$\alpha$ Absorbers (BLAs). BLAs are absorbers which have been thermally broadend to Doppler parameters in excess 40-45 km s$^{-1}$ due to high temperature which these BLAs trace. BLAs are expected to large reservoirs of baryons in the current universe, so we do a comprehensive survey of these BLAs to find the baryon content in them. 

In the first part of the work we study an interesting absorber system towards the line of sight of quasar PG 0003+158, which has a BLA candidate. We first fit the Voigt profiles to the absorption lines in the absorber system and then model the ionisation conditions in the absorber cloud to infer its ionisation state. Along with studying the galaxy neighbourhood of the absorber system we infer that the absorber is residing in a galaxy under-dense region and could be tracing a reservoir of baryons in a large scale filamentary structure in the cosmic web or CGM of a sub-$\text{L}^{*}$ galaxy.

The second part of the work includes our ambitious goal of survey the BLAs. For this BLA survey we found 29 suitable BLA candidates along 22 sight lines. The survey has a total \ion{H}{i} (Lyman-$\alpha$) unblocked redshift pathlength of $\Delta z = 5.561$. We perform Voigt profile measurements on these absorbers and also model the ionisation conditions prevailing in these systems. We identified absorption from 15 distinct metal ions apart from \ion{H}{i} absorption in our survey totalling to 711 absorption lines. We have modelled the ionisation conditions in 39 of the components in these 29 systems. Further, we use the results from the survey to estimate the baryon content of BLAs. We find that BLAs can contribute around 15-20 \% to the total cosmic baryon energy density and could be potential reservoirs of the  `missing baryons'.
