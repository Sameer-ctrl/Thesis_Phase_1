\chapter{Conclusions and Future Work} \label{ch:conclusions}

\section{Conclusions}

The motivation for the current work comes from `the missing baryon problem' in the current universe. According, to which around 30\% of the baryons are missing from our observations in the current universe. Cosmological simulations show that these baryons could be residing in Warm Hot Intergalactic Medium at high temperatures ($10^5-10^7$) K and very low densities ($10^{-4} - 10^{-6} \ \text{cm}^{-3}$). This makes it difficult to detect WHIM, which has lead to uncertainities in the baryon census in this phase. In the current work we aim to address these uncertainities by surveying the broad Lyman-$\alpha$ absorbers (BLAs), which are thermally broadend to large Doppler widths above 40-45 km s$^{-1}$. But detecting broad features does not mean that we are always probing WHIM or high temperature gas phase. Broad features could also arise due to contaminations from other lines, line blending effects, poor S/N in the spectra. To ensure that the detected broad feature is indeed tracing WHIM and not other case mentioned above, we need to model the ionisation conditions in these absorber systems. If we find that the species in the absorbers are coming from collisionally ionised phase, then we could be sure that BLA candidate is tracing a warm-hot plasma as collisional ionisation is prominentatly seen at high temperatures. One such specie is \ion{O}{vi}, which is an excellent tracer of WHIM, as it's ionisation fraction in collisional ionisation peaks around temperatures of $\text{T} \sim 10^{5.7}$ K.   

We use high S/N HST/COS FUV observations for the current work. In the low redshift IGM survey done by \citet{danforth-2016}, they have studied 82 quasar sight lines, which have the highest quality available data. They have done the preliminary Voigt profile measurements on the absorption lines along these 82 sight lines. But we need more accurate Voigt profile measurements for our survey. So, we look for suitable candidates for our survey in their dataset. We shortlist systems which have \ion{H}{i} Doppler parameters above 45 km s$^{-1}$ in their preliminary measurements. Since we also need to model the ionisation conditions in these absorbers, which require metal ions. So we further put constrain that the absorbers should have absorption from atleast 3 distinct metal ions to model the ionisation conditions robustly. We find 29 absorber systems which satisfies these two constrains. These 29 systems are distributed along 22 different sight lines. The survey has a total unblocked \ion{H}{i} (Lyman-$\alpha$) redshift pathlength of $\Delta z = 5.561$.  We find a total of 37 BLA components in these systems. Out of the 29 absorbers, 17 systems show absorption from \ion{O}{vi}. We have identified a total 413 absorption lines in these 29 systems and have modelled the ionisation conditions in 39 components of these systems. In case of \ion{O}{vi} systems, we find that 20 out of 25 \ion{O}{vi} components could be tracing collisionally ionised gas phase and possibly probing warm-hot plasma. 

We make use of the results from this survey to estimate the baryon content in BLAs. We consider three sample of absorbers for this. First, we use absorbers where reliable estimate of temperature is available and ionisation modelling suggests collisional ionisation of \ion{O}{vi}, this gives us a very conservative lower limit on $\Omega_b(\text{BLA})$ to be $(1.8 \pm 0.5) \times 10^{-3} {h_{70}}^{-1}$. Our second sample considers absorbers where \ion{O}{vi} is found to be collisionally ionised. For this, we estimate the temperature from Doppler parameters assuming complete thermal broadening. This gives us $\Omega_b(\text{BLA})=(7.2 \pm 1.3) \times 10^{-3} {h_{70}}^{-1}$. The last sample includes all the BLA components found in our survey. This yields a very large value of $\Omega_b(\text{BLA})=(27.1 \pm 13.8) \times 10^{-3} {h_{70}}^{-1}$, which exceeds the prediction of baryon content in WHIM from cosmological simulations of around 50\%. We discuss the sources on uncertainities in these estimates. We finally compare these values with the total baryon energy density and find that BLAs can contribute around 15-20 \% to the total cosmic baryon budget in the current universe.

\section{Future work}

The future work includes incorporating more sight lines in the surevy to improve the statistics. We can extend the results from this survey to larger data set of general population of BLA systems to get the more robust estimate of the baryon content in these BLAs. Future studies could be also be aimed to explore the correlations between these BLAs and galaxies by doing systematic survey of galaxies around these absorber system. This can allow us to understand the origin of these absorbers. This would allow us to get insights on the metal enrichment of the IGM from the galaxies. This could also shed light on the various feedback processes happening in the circumgalactic medium of the galaxies, which play crucial rule in the formation and evolution of the galaxies.  

