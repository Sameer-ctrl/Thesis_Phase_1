\documentclass[aspectratio=169]{beamer}
\usepackage[utf8]{inputenc}
\usepackage[T1]{fontenc}
\usepackage{mathabx}
\usepackage{mathpazo}
\usepackage{eulervm}
\usepackage{natbib}
\usepackage{fancyhdr}
\usepackage{tikz}
\usepackage{pgfpages}
\usepackage{pifont}


\makeatletter
\def\sectionentry#1#2#3#4#5{% section number, section title, page
%
\newcount\mymin%
\mymin=2
\ifnum\c@section=1%
    \mymin=3
\fi%
\ifnum\c@section=2%
    \mymin=2
\fi%
%
\newcount\mymax%
\mymax=2
\ifnum\c@section=\beamer@sectionmax%
    \mymax=3
\fi%
\ifnum\c@section=\numexpr\beamer@sectionmax-1%
    \mymax=2
\fi%
%
    \ifnum\numexpr\c@section-#1<\mymax%
        \ifnum\numexpr#1-\c@section<\mymin%
            \ifnum#5=\c@part%
                \beamer@section@set@min@width
                \box\beamer@sectionbox\hskip1.875ex plus 1fill%
                \beamer@xpos=0\relax%
                \beamer@ypos=1\relax%
                \setbox\beamer@sectionbox=
                \hbox{
                    \def\insertsectionhead{#2}%
                    \def\insertsectionheadnumber{#1}%
                    \def\insertpartheadnumber{#5}%

                    {%
                        \usebeamerfont{section in head/foot}\usebeamercolor[fg]{section in head/foot}%
                        \ifnum\c@section=#1%
                            \hyperlink{Navigation#3}{{\usebeamertemplate{section in head/foot}}}%
                        \else%
                            \hyperlink{Navigation#3}{{\usebeamertemplate{section in head/foot shaded}}}%
                        \fi%    
                    }%
                }%
                \ht\beamer@sectionbox=1.875ex%
                \dp\beamer@sectionbox=0.75ex%
            \fi%
        \fi%
    \fi%
    \ignorespaces%
}

\def\slideentry#1#2#3#4#5#6{%
	%section number, subsection number, slide number, first/last frame, page number, part number
	%
	\newcount\mymin%
	\mymin=2
	\ifnum\c@section=1%
		\mymin=3
	\fi%
	\ifnum\c@section=2%
		\mymin=2
	\fi%
		%
	\newcount\mymax%
	\mymax=2
	\ifnum\c@section=\beamer@sectionmax%
		\mymax=3
	\fi%
	\ifnum\c@section=\numexpr\beamer@sectionmax-1%
		\mymax=2
	\fi%
	%
	\ifnum\numexpr\c@section-#1<\mymax%
		\ifnum\numexpr#1-\c@section<\mymin%
		  \ifnum#6=\c@part\ifnum#2>0\ifnum#3>0%
		    \ifbeamer@compress%
		      \advance\beamer@xpos by1\relax%
		    \else%
		      \beamer@xpos=#3\relax%
		      \beamer@ypos=#2\relax%
		    \fi%
		  \hbox to 0pt{%
		    \beamer@tempdim=-\beamer@vboxoffset%
		    \advance\beamer@tempdim by-\beamer@boxsize%
		    \multiply\beamer@tempdim by\beamer@ypos%
		    \advance\beamer@tempdim by -.05cm%
		    \raise\beamer@tempdim\hbox{%
		      \beamer@tempdim=\beamer@boxsize%
		      \multiply\beamer@tempdim by\beamer@xpos%
		      \advance\beamer@tempdim by -\beamer@boxsize%
		      \advance\beamer@tempdim by 1pt%
		      \kern\beamer@tempdim
		      \global\beamer@section@min@dim\beamer@tempdim
		      \hbox{\beamer@link(#4){%
		          \usebeamerfont{mini frame}%
		          \ifnum\c@section=#1%
		            \ifnum\c@subsection=#2%
		              \usebeamercolor[fg]{mini frame}%
		              \ifnum\c@subsectionslide=#3%
		                \usebeamertemplate{mini frame}%\beamer@minislidehilight%
		              \else%
		                \usebeamertemplate{mini frame in current subsection}%\beamer@minisliderowhilight%
		              \fi%
		            \else%
		              \usebeamercolor{mini frame}%
		              %\color{fg!50!bg}%
		              \usebeamertemplate{mini frame in other subsection}%\beamer@minislide%
		            \fi%
		          \else%
		            \usebeamercolor{mini frame}%
		            %\color{fg!50!bg}%
		            \usebeamertemplate{mini frame in other subsection}%\beamer@minislide%
		          \fi%
		        }}}\hskip-10cm plus 1fil%
		  }\fi\fi%
		  \else%
		  \fakeslideentry{#1}{#2}{#3}{#4}{#5}{#6}%
		 \fi%
		\fi%
	\fi%
	\ignorespaces%
}
\makeatother




\setbeamertemplate{caption}[numbered]
\setbeamerfont{caption}{size=\scriptsize}
\setbeamercovered{transparent}


%% Load the markdown package
\usepackage[citations,footnotes,definitionLists,hashEnumerators,smartEllipses,tightLists=false,pipeTables,tableCaptions,hybrid]{markdown}
%%begin novalidate
\markdownSetup{rendererPrototypes={
 link = {\href{#2}{#1}},
 headingOne = {\section{#1}},
 headingTwo = {\subsection{#1}},
 headingThree = {\begin{frame}\frametitle{#1}},
 headingFour = {\begin{block}{#1}},
 horizontalRule = {\end{block}}
}}
%%end novalidate

\usetheme{Dresden}
\usefonttheme{serif}
\usecolortheme{rose}



\setbeamertemplate{headline}{%
\leavevmode%
  \hbox{%
    \begin{beamercolorbox}[wd=\paperwidth,ht=3.5ex,dp=2.75ex]{palette tertiary}%
    \scriptsize{\insertnavigation{\paperwidth}}
    \end{beamercolorbox}%
  }
  \hbox{%
    \begin{beamercolorbox}[wd=\paperwidth,ht=0.5ex,dp=1ex]{palette secondary}%
    \end{beamercolorbox}%
  }
}

\setbeamertemplate{footline}{%
\leavevmode%
    \hbox{%
    \begin{beamercolorbox}[wd=\paperwidth,ht=0.1ex,dp=1ex]{palette secondary}%
    \end{beamercolorbox}%
  }
  \hbox{%
    \begin{beamercolorbox}[wd=\paperwidth,ht=2.5ex,dp=1ex]{palette tertiary}%
     \hspace{2.5mm} \inserttitle \hfill{} \textbf{\insertframenumber} \hspace{2.5mm}
    \end{beamercolorbox}%
  } 
}



\author[Sameer Patidar]{\Large{Sameer Patidar}
\\ 
\vspace*{0.5mm}
SC19B161} % (optional, for multiple authors)
% {A.~B.~Arthur\inst{1} \and J.~Doe\inst{2}}

\institute[Indian Institute of Space Science and Technology] % (optional)
{
  \Large{Dual Degree (Astronomy and Astrophysics)} \\\\
\vspace{1mm}
  \Large{Indian Institute of Space Science and Technology}
}
\logo{\includegraphics[height=1cm]{IIST.png}}

\begin{document}

% TITLE OF THE PRESENTATION

\newcommand{\titleofpresentation}{Magnetic Fields and Relativistic Electrons fill entire Galaxy Cluster ?}

\date{\large{}}
\title{\LARGE{\textbf{\titleofpresentation}} 
\\
\vspace*{3mm}
\LARGE{\textbf{Seminar-II}}
}

\setbeamertemplate{headline}{%
\leavevmode%
  \hbox{%
    \begin{beamercolorbox}[wd=\paperwidth,ht=3.5ex,dp=2.75ex]{palette tertiary}%
    \end{beamercolorbox}%
  }
  \hbox{%
    \begin{beamercolorbox}[wd=\paperwidth,ht=0.5ex,dp=1ex]{palette secondary}%
    \end{beamercolorbox}%
  }
}

\setbeamertemplate{footline}{%
\leavevmode%
    \hbox{%
    \begin{beamercolorbox}[wd=\paperwidth,ht=0.1ex,dp=1ex]{palette secondary}%
    \end{beamercolorbox}%
  }
  \hbox{%
    \begin{beamercolorbox}[wd=\paperwidth,ht=2.5ex,dp=1ex]{palette tertiary}%
    \end{beamercolorbox}%
  }
  
}

\begin{frame}
\maketitle
\end{frame}

\setbeamertemplate{footline}{%
\leavevmode%
    \hbox{%
    \begin{beamercolorbox}[wd=\paperwidth,ht=0.1ex,dp=1ex]{palette secondary}%
    \end{beamercolorbox}%
  }
  \hbox{%
    \begin{beamercolorbox}[wd=\paperwidth,ht=2.5ex,dp=1ex]{palette tertiary}%
     \hspace{2.5mm} \textbf{\titleofpresentation} \hfill{} \textbf{\insertframenumber} \hspace{2.5mm}
    \end{beamercolorbox}%
  } 
}


\logo{}

\begin{frame}
\frametitle{\huge{{\textbf{Outline}}}}
\tableofcontents
\end{frame}

% -----------------------------------------------------------------------------------------------------
% -----------------------------------------------------------------------------------------------------
% -----------------------------------------------------------------------------------------------------

\setbeamertemplate{headline}{%
\leavevmode%
  \hbox{%
    \begin{beamercolorbox}[wd=\paperwidth,ht=3.5ex,dp=2.75ex]{palette tertiary}%
    \scriptsize{\textbf{\insertnavigation{\paperwidth}}}
    \end{beamercolorbox}%
  }
  \hbox{%
    \begin{beamercolorbox}[wd=\paperwidth,ht=0.5ex,dp=1ex]{palette secondary}%
    \end{beamercolorbox}%
  }
}

\setbeamertemplate{footline}{%
\leavevmode%
    \hbox{%
    \begin{beamercolorbox}[wd=\paperwidth,ht=0.1ex,dp=1ex]{palette secondary}%
    \end{beamercolorbox}%
  }
  \hbox{%
    \begin{beamercolorbox}[wd=\paperwidth,ht=2.5ex,dp=1ex]{palette tertiary}%
     \hspace{2.5mm} \textbf{\titleofpresentation} \hfill{} \textbf{\insertframenumber} \hspace{2.5mm}
    \end{beamercolorbox}%
  } 
}


\begin{markdown}

% -----------------------------------------------------------------------------------------------------
% -----------------------------------------------------------------------------------------------------
% -----------------------------------------------------------------------------------------------------

# Seminar-I : Overview

\vspace*{2.7cm}

{\huge{\textbf{Seminar-I : Overview}}}

\end{frame}

% -----------------------------------------------------------------------------------------------------
% -----------------------------------------------------------------------------------------------------
% -----------------------------------------------------------------------------------------------------

### \huge{{\textbf{Seminar-I : Overview}}}

\begin{itemize}
    \item[-] Galaxy Clusters
    \item[-] Emissions from Galaxy Clusters
    \item[-] Diffuse Radio Emission from Galaxy Clusters
    \item[-] Acceleration Mechanisms
    \item[-] Classification of Cluster Radio Sources
    \item[-] Cluster Magnetic Fields
\end{itemize}


\end{frame}

% -----------------------------------------------------------------------------------------------------
% -----------------------------------------------------------------------------------------------------
% -----------------------------------------------------------------------------------------------------

# Motivation

\vspace*{2.7cm}

{\huge{\textbf{Motivation}}}

\end{frame}

% -----------------------------------------------------------------------------------------------------
% -----------------------------------------------------------------------------------------------------
% -----------------------------------------------------------------------------------------------------

### \huge{{\textbf{Motivation}}}


\begin{itemize}
            \uncover<1->{\item Diffuse synchrotron emission in disturbed galaxy clusters}
            \begin{itemize}
                \uncover<2->{\item[-] Radio sources : central halos and peripheral relics}
                \uncover<3->{\item[-] Shocks and turbulence can accelerate particles to relativistic energies and amplify the magnetic fields in central cubic Mpc.} 
            \end{itemize}
\end{itemize}

\begin{itemize}
    \uncover<4->{\item Cosmological simulations predicts that these mechanisms extend to much larger scales}
\end{itemize}
             
            
\end{frame}

% -----------------------------------------------------------------------------------------------------
% -----------------------------------------------------------------------------------------------------
% -----------------------------------------------------------------------------------------------------

# Observations


\begin{columns}
    \begin{column}{0.4\textwidth} 
        \vspace{1mm}
        \huge{\textbf{Observations}} 
    \end{column}
    \begin{column}{0.6\textwidth}
        \vspace{8mm}
        \begin{figure}[!htbp]
          \centering
          \includegraphics[width=6cm]{Figures/Seminar-2/LOFAR_Superterp.jpg}
          \vspace*{-1mm}
          \caption{The LOFAR 'superterp', part of the core of the extended telescope located in the Netherlands. Credits : LOFAR/ASTRON.}
          \label{}
        \end{figure}
    \end{column}
\end{columns}

\end{frame}

% -----------------------------------------------------------------------------------------------------
% -----------------------------------------------------------------------------------------------------
% -----------------------------------------------------------------------------------------------------

### \huge{{\textbf{Observations}}}


\begin{columns}
    \begin{column}{0.6\textwidth} 
         \begin{itemize}
             \item Observed galaxy cluster Abell 2255 (z=0.08)
             \item Deep observations at 49 and 145 MHz with LOFAR 
                \begin{itemize}
                    \item[-] Low Band Antenna (LBA)
                    \item[-] High Band Antenna (HBA)
                \end{itemize} 
            \item 145 MHz : $\times 60$ lower noise and $\times 25$ angular resolution than WSRT at 150 Mhz
        \end{itemize}
    \end{column}
    \begin{column}{0.4\textwidth}
        \begin{figure}[!htbp]
          \centering
          \includegraphics[width=4cm]{Figures/Seminar-2/Abell2255_1000.jpeg}
          \vspace*{-1mm}
          \caption{SDSS image of Abell 2255 cluster. Credits : Sloan Digital Sky Survey V}
          \label{}
        \end{figure}
    \end{column}
\end{columns}
             
            
\end{frame}


% -----------------------------------------------------------------------------------------------------
% -----------------------------------------------------------------------------------------------------
% -----------------------------------------------------------------------------------------------------

# Results from LOFAR

\begin{frame}{}

\vspace*{1.5cm}

{\huge{\textbf{Results from LOFAR}}}

\begin{itemize}
    \item[-] \Large{Images}
    \item[-] \Large{Spectral Maps}
\end{itemize}

\end{frame}

% -----------------------------------------------------------------------------------------------------
% -----------------------------------------------------------------------------------------------------
% -----------------------------------------------------------------------------------------------------

\begin{frame}[plain]{}

% \uncover<1->{\item Diffuse synchrotron emission in disturbed galaxy clusters}

\begin{columns}
    \begin{column}{0.3\textwidth} 
         \begin{itemize}
             \item {\textbf{Central region}}
                \begin{itemize}
                    \item[-] Halo
                    \item[-] Relic
                    \item[-] AGNs
                \end{itemize} 
             \item Envelope
             \item Relics in peripheral regions 
             \\\\ \vspace{16mm} \hspace{-4mm}
             \scriptsize{
             Yellow : 145 MHz - High Res. \\\\ \hspace{-3mm}
             Magenta : 49 MHz - Low Res.\\\\ \hspace{-3mm}
             Blue : X-ray (0.1 to 2.4 keV)}
        \end{itemize}
    \end{column}
    \begin{column}{0.8\textwidth}
         \vspace{-2mm}
          \begin{figure}[!htbp]
          \centering
          \includegraphics[width=8cm]{Figures/Seminar-2/Composite image.jpg}
          \vspace*{-3mm}
          \caption{Composite image of Abell 2255 cluster. Credits :  A. Botteon et al. (2022)}
          \label{}
        \end{figure}
    \end{column}
\end{columns}

    
\end{frame}

% -----------------------------------------------------------------------------------------------------
% -----------------------------------------------------------------------------------------------------
% -----------------------------------------------------------------------------------------------------

\begin{frame}[plain,noframenumbering]{}


\begin{columns}
    \begin{column}{0.3\textwidth} 
         \begin{itemize}
             \item {\textbf{Central region}}
                \begin{itemize}
                    \item[-] Halo
                    \item[-] Relic
                    \item[-] AGNs
                \end{itemize} 
             \item Envelope
             \item Relics in peripheral regions 
             \\\\ \vspace{16mm} \hspace{-4mm}
             \scriptsize{
             Yellow : 145 MHz - High Res. \\\\ \hspace{-3mm}
             Magenta : 49 MHz - Low Res.\\\\ \hspace{-3mm}
             Blue : X-ray (0.1 to 2.4 keV)}
        \end{itemize}
    \end{column}
    \begin{column}{0.8\textwidth}
         \vspace{-2mm}
          \begin{figure}[!htbp]
          \centering
          \includegraphics[width=11cm]{Figures/Seminar-2/High res.png}
          \vspace*{-1mm}
          \caption{Highest-resolution LOFAR images (a) LBA 49 MHz  (11.5" $\times$ 8:2") \& (b) HBA 145 MHz (4.7" $\times$ 3.5") . Credits :  A. Botteon et al. (2022)}
          \label{}
        \end{figure}
    \end{column}
\end{columns}

\end{frame}

% -----------------------------------------------------------------------------------------------------
% -----------------------------------------------------------------------------------------------------
% -----------------------------------------------------------------------------------------------------

\begin{frame}[plain,noframenumbering]{}

\begin{columns}
    \begin{column}{0.3\textwidth} 
         \begin{itemize}
             \item Central region
                \begin{itemize}
                    \item[-] Halo
                    \item[-] Relic
                    \item[-] AGNs
                \end{itemize} 
             \item {\textbf{Envelope}}
             \item Relics in peripheral regions 
             \\\\ \vspace{16mm} \hspace{-4mm}
             \scriptsize{
             Yellow : 145 MHz - High Res. \\\\ \hspace{-3mm}
             Magenta : 49 MHz - Low Res.\\\\ \hspace{-3mm}
             Blue : X-ray (0.1 to 2.4 keV)}
        \end{itemize}
    \end{column}
    \begin{column}{0.8\textwidth}
         \vspace{-2mm}
          \begin{figure}[!htbp]
          \centering
          \includegraphics[width=8cm]{Figures/Seminar-2/Composite image.jpg}
          \vspace*{-3mm}
          \caption{Composite image of Abell 2255 cluster. Credits :  A. Botteon et al. (2022)}
          \label{}
        \end{figure}
    \end{column}
\end{columns}

    
\end{frame}

% -----------------------------------------------------------------------------------------------------
% -----------------------------------------------------------------------------------------------------
% -----------------------------------------------------------------------------------------------------

\begin{frame}[plain,noframenumbering]{}


\begin{columns}
    \begin{column}{0.3\textwidth} 
         \begin{itemize}
             \item Central region
                \begin{itemize}
                    \item[-] Halo
                    \item[-] Relic
                    \item[-] AGNs
                \end{itemize} 
             \item {\textbf{Envelope}}
             \item Relics in peripheral regions 
             \\\\ \vspace{16mm} \hspace{-4mm}
             \scriptsize{
             Yellow : 145 MHz - High Res. \\\\ \hspace{-3mm}
             Magenta : 49 MHz - Low Res.\\\\ \hspace{-3mm}
             Blue : X-ray (0.1 to 2.4 keV)}
        \end{itemize}
    \end{column}
    \begin{column}{0.8\textwidth}
         \vspace{-2mm}
          \begin{figure}[!htbp]
          \centering
          \includegraphics[width=8cm]{Figures/Seminar-2/49 Mhz low.png}
          \vspace*{-3mm}
          \caption{LOFAR LBA 49 MHz radio image. Resolution : 20.0" $\times$ 12.2". Credits :  A. Botteon et al. (2022)}
          \label{}
        \end{figure}
    \end{column}
\end{columns}

\end{frame}

% -----------------------------------------------------------------------------------------------------
% -----------------------------------------------------------------------------------------------------
% -----------------------------------------------------------------------------------------------------

\begin{frame}[plain,noframenumbering]{}

\begin{columns}
    \begin{column}{0.31\textwidth} 
         \begin{itemize}
             \item Central region
                \begin{itemize}
                    \item[-] Halo
                    \item[-] Relic
                    \item[-] AGNs
                \end{itemize} 
             \item Envelope
             \item {\textbf{Relics in peripheral regions}} 
             \\\\ \vspace{16mm} \hspace{-4mm}
             \scriptsize{
             Yellow : 145 MHz - High Res. \\\\ \hspace{-3mm}
             Magenta : 49 MHz - Low Res.\\\\ \hspace{-3mm}
             Blue : X-ray (0.1 to 2.4 keV)}
        \end{itemize}
    \end{column}
    \begin{column}{0.8\textwidth}
         \vspace{-2mm}
          \begin{figure}[!htbp]
          \centering
          \includegraphics[width=8cm]{Figures/Seminar-2/Composite image.jpg}
          \vspace*{-3mm}
          \caption{Composite image of Abell 2255 cluster. Credits :  A. Botteon et al. (2022)}
          \label{}
        \end{figure}
    \end{column}
\end{columns}

    
\end{frame}

% -----------------------------------------------------------------------------------------------------
% -----------------------------------------------------------------------------------------------------
% -----------------------------------------------------------------------------------------------------

\begin{frame}{\huge{\textbf{Spectral Index Maps}}}

\begin{columns}
    \begin{column}{0.6\textwidth} 
         \begin{itemize}
             \item Computed spectral index on pixel basis $^{\*}$
             \item Surface brightness threshold of $2\sigma$ or $3\sigma$ 
             \item Different spectral maps :
                \begin{itemize}
                    \item[-] At 12.5" with discrete sources
                    \item[-] At 60" with discrete sources subtracted
                    \item[-] At 35" with discrete sources subtracted $^{\*}$
                \end{itemize} 
        \end{itemize}
    \end{column}
    \begin{column}{0.45\textwidth}
    \begin{center}
        \includegraphics[width=6cm]{Figures/Seminar-2/spec_index.png}
    \end{center}
    \end{column}
\end{columns}

\end{frame}

% -----------------------------------------------------------------------------------------------------
% -----------------------------------------------------------------------------------------------------
% -----------------------------------------------------------------------------------------------------

\begin{frame}[plain]


\begin{tikzpicture}[remember picture, overlay]
\vspace{-10mm}
\node[above=-81mm] at (current page.north) (A)
{ 
    \includegraphics[height=7.8cm]{Figures/Seminar-2/Spec index map 1-2.png}
};
\end{tikzpicture}

\begin{figure}
    \centering
    \includegraphics{}
    \vspace{71mm}
    \caption{Spectral index maps. (a) and (b) are at common resolution of 60" and 35" respectively with discrete sources subtracted. Credits : A. Botteon et al. (2022)}
    \label{}
\end{figure}
    
\end{frame}

% -----------------------------------------------------------------------------------------------------
% -----------------------------------------------------------------------------------------------------
% -----------------------------------------------------------------------------------------------------

\begin{frame}[plain]


\begin{tikzpicture}[remember picture, overlay]
\vspace{-10mm}
\node[above=-75mm] at (current page.north) (A)
{ 
    \includegraphics[height=7cm]{Figures/Seminar-2/Spec index map 1.png}
};
\end{tikzpicture}

\begin{figure}
    \centering
    \includegraphics{}
    \vspace{67mm}
    \caption{Images and spectral index map. The images have a common resolution of 12.5" with discrete sources. Credits : A. Botteon et al. (2022)}
    \label{}
\end{figure}
    
\end{frame}

% -----------------------------------------------------------------------------------------------------
% -----------------------------------------------------------------------------------------------------
% -----------------------------------------------------------------------------------------------------

### \huge{\textbf{Spectral Index Maps}}

\begin{columns}
    \begin{column}{0.6\textwidth} 
         \begin{itemize}
             \item Different peaks
             \item Flat-spectrum : acceleration of $\text{e}^-$ near shock fronts
             \item Steeper spectrum : acceleration by turbulence
        \end{itemize}
    \end{column}
    \begin{column}{0.4\textwidth}
    \begin{figure}
        \centering
        \includegraphics[width=6cm]{Figures/Seminar-2/Spec index histrogram.png}
        \vspace{-6mm}
        \caption{Spectral index histogram. Credits : A. Botteon et al. (2022)}
        \label{}
    \end{figure}
    \end{column}
\end{columns}

\end{frame}

% -----------------------------------------------------------------------------------------------------
% -----------------------------------------------------------------------------------------------------
% -----------------------------------------------------------------------------------------------------

# Magnetic Fields

\begin{frame}{}

\vspace*{1.5cm}

{\huge{\textbf{Magnetic Fields}}}
\\\\
\begin{itemize}
    \item[-] \Large{Constraints}
    \item[-] \Large{Origin}
\end{itemize}

\end{frame}

% -----------------------------------------------------------------------------------------------------
% -----------------------------------------------------------------------------------------------------
% -----------------------------------------------------------------------------------------------------



### \huge{\textbf{Constraints}}

\begin{itemize}
              \uncover<1->{\item Magnetic field can be estimated from observed synchrotron radiation}
              \uncover<1->{\item 
              \begin{tikzpicture}[remember picture, overlay]
                    \vspace{-10mm}
                    \node[above=-45mm,left=-27mm] at (current page.north) (A)
                    { 
                        \includegraphics[width=9cm]{Figures/Seminar-2/B_constraint.png}
                    };
             \end{tikzpicture}}
              \uncover<3->{\item Lower limit of few 0.1$\mu$G}  
\end{itemize}



\end{frame}

% -----------------------------------------------------------------------------------------------------
% -----------------------------------------------------------------------------------------------------
% -----------------------------------------------------------------------------------------------------


### \huge{\textbf{Origin}}

\begin{itemize}
              \uncover<1->{\item Extragalactic magnetic fields may have primordial origin}
              \begin{itemize}
                  \uncover<2->{\item[-] Lower limit derived $\sim$ 250 times larger than upper limit of such field} 
                  \uncover<3->{\item[-] Atleast an order of magnitude larger than value expected from compression of such field}
              \end{itemize}
              \uncover<4->{\item $\mu$G fields can be produced from small-scale turbulent dynamo amplification in central regions} 
\end{itemize}



\end{frame}

% -----------------------------------------------------------------------------------------------------
% -----------------------------------------------------------------------------------------------------
% -----------------------------------------------------------------------------------------------------


# Origin of Observed Radio Emission

\vspace*{2.7cm}

{\huge{\textbf{Origin of Observed Radio Emission}}}

\end{frame}


% -----------------------------------------------------------------------------------------------------
% -----------------------------------------------------------------------------------------------------
% -----------------------------------------------------------------------------------------------------


### {\huge{\textbf{Origin of Observed Radio Emission}}}

\begin{itemize}
             \uncover<1->{\item Sources of relativistic particles ?}
             \begin{itemize}
                  \uncover<2->{\item[-] Accelerated from thermal matter ?}
                  \uncover<3->{\item[-] Accelerated from preexisting suprathermal particles ?} 
             \end{itemize} 
             \uncover<4->{\item Energy of thermal matter $\sim$ keVs}
             \uncover<5->{\item Minimum KE of $\text{e}^-$ : 20-30 MeV}
             \uncover<6->{\item Spectrum can't extend at energies $\leq$ several MeV}
            
\end{itemize}


\end{frame}


% -----------------------------------------------------------------------------------------------------
% -----------------------------------------------------------------------------------------------------
% -----------------------------------------------------------------------------------------------------


### {\huge{\textbf{Origin of Observed Radio Emission}}}

\begin{itemize}
             \uncover<1->{\item \textbf{Electrons are accelerated from preexisting pool of suprathermal particles}}
             \uncover<2->{\item Cooling time of e$^-$ in galaxy clusters :
                            \begin{itemize}
                                \item[-] Coulumb losses at lower energies
                                \item[-] Radiative losses at higher energies
                            \end{itemize}
                         }
             \uncover<3->{\item These e$^-$ can get accumulated in ICM}
\end{itemize}


\end{frame}


% -----------------------------------------------------------------------------------------------------
% -----------------------------------------------------------------------------------------------------
% -----------------------------------------------------------------------------------------------------


### {\huge{\textbf{Origin of Observed Radio Emission}}}

\begin{itemize}
             \uncover<1->{\item Sources of seed e$^-$ :}
                        \begin{itemize}
                            \uncover<2->{\item[-] Shocks during cluster's assembly}
                            \uncover<3->{\item[-] Individual galaxies :}
                            \begin{itemize}
                                \uncover<4->{\item[\ding{226}] Through AGN outflows} 
                                \uncover<5->{\item[\ding{226}] Star formation activities like stellar winds or SNe} 
                            \end{itemize}
                        \end{itemize}
             \uncover<6->{\item Outermost diffuse radio emission : Products of shock waves}
             \uncover<7->{\item These can produce emission via reacceleration mechanisms or adiabatic compression processes}
\end{itemize}


\end{frame}


% -----------------------------------------------------------------------------------------------------
% -----------------------------------------------------------------------------------------------------
% -----------------------------------------------------------------------------------------------------


### {\huge{\textbf{Acceleration Efficiency}}}

\begin{center}
    \includegraphics[width=7cm]{Figures/Seminar-2/efficiency.png}
\end{center}

- To get observed radio emission, 
\begin{tikzpicture}[remember picture, overlay]
                    \vspace{-10mm}
                    \node[above=-66mm,left=-35mm] at (current page.north) (A)
                    { 
                        \includegraphics[width=4cm]{Figures/Seminar-2/efficeincy_Sum.png}
                    };
\end{tikzpicture}
- Efficiency similar to Magneto-Hydrodynamics turbulence



\end{frame}


% -----------------------------------------------------------------------------------------------------
% -----------------------------------------------------------------------------------------------------
% -----------------------------------------------------------------------------------------------------




# Numerical Simulations

\vspace*{2.7cm}

{\huge{\textbf{Numerical Simulations}}}

\end{frame}


% -----------------------------------------------------------------------------------------------------
% -----------------------------------------------------------------------------------------------------
% -----------------------------------------------------------------------------------------------------

### {\huge{\textbf{Numerical Simulations}}}


- z=0.02 snapshot of high resolution cosmological simulation of massive galaxy cluster was analysed
- Simulated cluster : $M_{200}=8.65\times {10}^{14} \ M_\odot$  within $r_{200}=1.96$ Mpc
- Abell 2255 : $M_{200}=10.33\times {10}^{14} \ M_\odot$  within $r_{200}=2.03$ Mpc



\end{frame}


% -----------------------------------------------------------------------------------------------------
% -----------------------------------------------------------------------------------------------------
% -----------------------------------------------------------------------------------------------------


# Conclusions

\vspace*{2.7cm}

{\huge{\textbf{Conclusions}}}

\end{frame}


% -----------------------------------------------------------------------------------------------------
% -----------------------------------------------------------------------------------------------------
% -----------------------------------------------------------------------------------------------------

### {\huge{\textbf{Conclusions}}}

\begin{itemize}
        \uncover<1->{
                \item Cosmic rays and Magnetic fields exist at large distances from cluster centre
                \item Energy is efficiently converted into non-thermal components
                \item Energy budget of non-thermal components and the magnetic fields is constrained
                }
            \\\\ \vspace{5mm}
        \uncover<2->{\textbf{\LARGE{So, magnetic fields and relativistic electrons do fill an entire galaxy cluster
}}}
\end{itemize}


\end{frame}



% -----------------------------------------------------------------------------------------------------
% -----------------------------------------------------------------------------------------------------
% -----------------------------------------------------------------------------------------------------

\begin{frame}<0>[noframenumbering]

{\cite{Andrea} ,\cite{Botteon_2020}, \cite{},\cite{Weeren},\cite{Brunetti}}
    
\end{frame}


% -----------------------------------------------------------------------------------------------------
% -----------------------------------------------------------------------------------------------------
% -----------------------------------------------------------------------------------------------------


\begin{frame}{\huge{\textbf{References}}}
\renewcommand{\bibfont}{\footnotesize}
% \frametitle{\huge{\textbf{References}}}

\bibliographystyle{myunsrt}
\bibliography{refs}

\end{frame}

% -----------------------------------------------------------------------------------------------------
% -----------------------------------------------------------------------------------------------------
% -----------------------------------------------------------------------------------------------------

# 


\begin{frame}{}
  \centering \Huge
  \textbf{\emph{Such good times...}}
\end{frame}
% In the End, we are Cosmic Dust...

% -----------------------------------------------------------------------------------------------------
% -----------------------------------------------------------------------------------------------------
% -----------------------------------------------------------------------------------------------------

\end{markdown}

\end{document}



% ---------------Column-----------------

% \begin{columns}
%     \begin{column}{0.6\textwidth} 
%          \begin{itemize}
%              \uncover<1->{\item Largest virialised objects in the universe} 
%             \uncover<2->{\item Masses upto $\sim 10^{16} M_{\odot}$ }
%             \uncover<3->{\item ICM - $10^7-10^8 \ \ K$} 
%             \uncover<4->{\item Elongated filaments of galaxies span region between clusters}
%             \uncover<5->{\item WHIM pervades these filaments} 
%             \uncover<6->{\item Dynamical state - Relaxed and Merging clusters} 
%         \end{itemize}
%     \end{column}
%     \begin{column}{0.4\textwidth}
%         \begin{figure}[!htbp]
%           \centering
%           \includegraphics[width=5cm]{Figures/Cluster MACS J1206.2-0847.jpg}
%           \vspace*{-1mm}
%           \caption{HST image showing the galaxy cluster MACS J1206. Credits: NASA, ESA, M. Postman (STScI) and the CLASH Team}
%           \label{}
%         \end{figure}
%     \end{column}
% \end{columns}


% -----------Figure Placing------------

% \begin{tikzpicture}[remember picture, overlay]
% \node[right=-1mm] at (current page.west) (A)
% {
%     \includegraphics[width=16cm]{Figures/Abell 2744 Emission.png}
%     \vspace*{-1mm}
% };
% \end{tikzpicture}

% \begin{figure}
%     \centering
%     \includegraphics{}
%     \vspace*{50mm}
%     \caption{The galaxy cluster Abell 2744. Credits : Medezinski et al. (2016), Merten et al. (2011), Lotz et al. (2017), R. J. van Weeren et al. (2019).}
%     \label{}
% \end{figure}


























% ### Projected Profit

% 1. And the answer is...
% 2. $f(x)=\sum_{n=0}^\infty\frac{f^{(n)}(a)}{n!}(x-a)^n$
%     #. How do we _know_ that?
%     #. __Maths!__

% \end{frame}

% ### Testing blocks

% #### This is a block!

% - Here is some content.
% - Here's more contents.

% ---

% \end{frame}


% ### Citations

% - This is a citation [@novotny:2017]
% - Works like `natbib` syntax [see @novotny:2019 p.26]

% \end{frame}


% ### Pipe Tables

% - Use `pipeTables` and `tableCaptions` options
% - Available since `markdown` v2.8.0

% | Right | Left | Default | Center |
% |------:|:-----|---------|:------:| 
% |  12   |  12  |  12     |   12   | 
% | 123   |  123 |   123   |  123   | 
% |   1   |    1 |     1   |    1   | 

%   : Demonstration of pipe table syntax.
  
% \end{frame}

% %%novalidate

% \end{markdown}

% \begin{frame}
% \renewcommand{\bibfont}{\footnotesize}
% \frametitle{Bibliography}

% \bibliographystyle{apalike}
% \bibliography{refs}

% \end{frame}



% \end{document}
